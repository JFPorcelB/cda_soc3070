% Options for packages loaded elsewhere
\PassOptionsToPackage{unicode}{hyperref}
\PassOptionsToPackage{hyphens}{url}
%
\documentclass[
  8pt,
  ignorenonframetext,
]{beamer}
\usepackage{pgfpages}
\setbeamertemplate{caption}[numbered]
\setbeamertemplate{caption label separator}{: }
\setbeamercolor{caption name}{fg=normal text.fg}
\beamertemplatenavigationsymbolsempty
% Prevent slide breaks in the middle of a paragraph
\widowpenalties 1 10000
\raggedbottom
\setbeamertemplate{part page}{
  \centering
  \begin{beamercolorbox}[sep=16pt,center]{part title}
    \usebeamerfont{part title}\insertpart\par
  \end{beamercolorbox}
}
\setbeamertemplate{section page}{
  \centering
  \begin{beamercolorbox}[sep=12pt,center]{part title}
    \usebeamerfont{section title}\insertsection\par
  \end{beamercolorbox}
}
\setbeamertemplate{subsection page}{
  \centering
  \begin{beamercolorbox}[sep=8pt,center]{part title}
    \usebeamerfont{subsection title}\insertsubsection\par
  \end{beamercolorbox}
}
\AtBeginPart{
  \frame{\partpage}
}
\AtBeginSection{
  \ifbibliography
  \else
    \frame{\sectionpage}
  \fi
}
\AtBeginSubsection{
  \frame{\subsectionpage}
}
\usepackage{amsmath,amssymb}
\usepackage{lmodern}
\usepackage{iftex}
\ifPDFTeX
  \usepackage[T1]{fontenc}
  \usepackage[utf8]{inputenc}
  \usepackage{textcomp} % provide euro and other symbols
\else % if luatex or xetex
  \usepackage{unicode-math}
  \defaultfontfeatures{Scale=MatchLowercase}
  \defaultfontfeatures[\rmfamily]{Ligatures=TeX,Scale=1}
\fi
\usetheme[]{Copenhagen}
\usecolortheme{dolphin}
\usefonttheme{structurebold}
% Use upquote if available, for straight quotes in verbatim environments
\IfFileExists{upquote.sty}{\usepackage{upquote}}{}
\IfFileExists{microtype.sty}{% use microtype if available
  \usepackage[]{microtype}
  \UseMicrotypeSet[protrusion]{basicmath} % disable protrusion for tt fonts
}{}
\makeatletter
\@ifundefined{KOMAClassName}{% if non-KOMA class
  \IfFileExists{parskip.sty}{%
    \usepackage{parskip}
  }{% else
    \setlength{\parindent}{0pt}
    \setlength{\parskip}{6pt plus 2pt minus 1pt}}
}{% if KOMA class
  \KOMAoptions{parskip=half}}
\makeatother
\usepackage{xcolor}
\newif\ifbibliography
\usepackage{color}
\usepackage{fancyvrb}
\newcommand{\VerbBar}{|}
\newcommand{\VERB}{\Verb[commandchars=\\\{\}]}
\DefineVerbatimEnvironment{Highlighting}{Verbatim}{commandchars=\\\{\}}
% Add ',fontsize=\small' for more characters per line
\usepackage{framed}
\definecolor{shadecolor}{RGB}{248,248,248}
\newenvironment{Shaded}{\begin{snugshade}}{\end{snugshade}}
\newcommand{\AlertTok}[1]{\textcolor[rgb]{0.94,0.16,0.16}{#1}}
\newcommand{\AnnotationTok}[1]{\textcolor[rgb]{0.56,0.35,0.01}{\textbf{\textit{#1}}}}
\newcommand{\AttributeTok}[1]{\textcolor[rgb]{0.77,0.63,0.00}{#1}}
\newcommand{\BaseNTok}[1]{\textcolor[rgb]{0.00,0.00,0.81}{#1}}
\newcommand{\BuiltInTok}[1]{#1}
\newcommand{\CharTok}[1]{\textcolor[rgb]{0.31,0.60,0.02}{#1}}
\newcommand{\CommentTok}[1]{\textcolor[rgb]{0.56,0.35,0.01}{\textit{#1}}}
\newcommand{\CommentVarTok}[1]{\textcolor[rgb]{0.56,0.35,0.01}{\textbf{\textit{#1}}}}
\newcommand{\ConstantTok}[1]{\textcolor[rgb]{0.00,0.00,0.00}{#1}}
\newcommand{\ControlFlowTok}[1]{\textcolor[rgb]{0.13,0.29,0.53}{\textbf{#1}}}
\newcommand{\DataTypeTok}[1]{\textcolor[rgb]{0.13,0.29,0.53}{#1}}
\newcommand{\DecValTok}[1]{\textcolor[rgb]{0.00,0.00,0.81}{#1}}
\newcommand{\DocumentationTok}[1]{\textcolor[rgb]{0.56,0.35,0.01}{\textbf{\textit{#1}}}}
\newcommand{\ErrorTok}[1]{\textcolor[rgb]{0.64,0.00,0.00}{\textbf{#1}}}
\newcommand{\ExtensionTok}[1]{#1}
\newcommand{\FloatTok}[1]{\textcolor[rgb]{0.00,0.00,0.81}{#1}}
\newcommand{\FunctionTok}[1]{\textcolor[rgb]{0.00,0.00,0.00}{#1}}
\newcommand{\ImportTok}[1]{#1}
\newcommand{\InformationTok}[1]{\textcolor[rgb]{0.56,0.35,0.01}{\textbf{\textit{#1}}}}
\newcommand{\KeywordTok}[1]{\textcolor[rgb]{0.13,0.29,0.53}{\textbf{#1}}}
\newcommand{\NormalTok}[1]{#1}
\newcommand{\OperatorTok}[1]{\textcolor[rgb]{0.81,0.36,0.00}{\textbf{#1}}}
\newcommand{\OtherTok}[1]{\textcolor[rgb]{0.56,0.35,0.01}{#1}}
\newcommand{\PreprocessorTok}[1]{\textcolor[rgb]{0.56,0.35,0.01}{\textit{#1}}}
\newcommand{\RegionMarkerTok}[1]{#1}
\newcommand{\SpecialCharTok}[1]{\textcolor[rgb]{0.00,0.00,0.00}{#1}}
\newcommand{\SpecialStringTok}[1]{\textcolor[rgb]{0.31,0.60,0.02}{#1}}
\newcommand{\StringTok}[1]{\textcolor[rgb]{0.31,0.60,0.02}{#1}}
\newcommand{\VariableTok}[1]{\textcolor[rgb]{0.00,0.00,0.00}{#1}}
\newcommand{\VerbatimStringTok}[1]{\textcolor[rgb]{0.31,0.60,0.02}{#1}}
\newcommand{\WarningTok}[1]{\textcolor[rgb]{0.56,0.35,0.01}{\textbf{\textit{#1}}}}
\setlength{\emergencystretch}{3em} % prevent overfull lines
\providecommand{\tightlist}{%
  \setlength{\itemsep}{0pt}\setlength{\parskip}{0pt}}
\setcounter{secnumdepth}{-\maxdimen} % remove section numbering
\ifLuaTeX
  \usepackage{selnolig}  % disable illegal ligatures
\fi
\IfFileExists{bookmark.sty}{\usepackage{bookmark}}{\usepackage{hyperref}}
\IfFileExists{xurl.sty}{\usepackage{xurl}}{} % add URL line breaks if available
\urlstyle{same} % disable monospaced font for URLs
\hypersetup{
  pdftitle={Análisis de Datos Categóricos},
  pdfauthor={Ayudantía 4},
  hidelinks,
  pdfcreator={LaTeX via pandoc}}

\title{Análisis de Datos Categóricos}
\author{Ayudantía 4}
\date{Felipe Olivares}

\begin{document}
\frame{\titlepage}

\begin{frame}{Contenido}
\protect\hypertarget{contenido}{}
\begin{enumerate}
\item
  Loops en R
\item
  Medidas de asociación
\end{enumerate}
\end{frame}

\begin{frame}[fragile]{Loops en R}
\protect\hypertarget{loops-en-r}{}
Los loops (``ciclos'' o ``bucles'' en español) son un tipo especial de
funciones en \texttt{R} que sirven para ejecutar una tarea determinada
una cantidad \emph{n} de veces. Se llama iteración a cada una de estas
repeticiones, y sirven para hacer en segundos lo que manualmente
llevaría horas, días o sería simplemente demasiado.

\begin{Shaded}
\begin{Highlighting}[]
\CommentTok{\# Sintaxis for}

\ControlFlowTok{for}\NormalTok{ (i }\ControlFlowTok{in} \DecValTok{1}\SpecialCharTok{:}\DecValTok{4}\NormalTok{)\{}
  \FunctionTok{print}\NormalTok{(i)}
\NormalTok{\}}
\end{Highlighting}
\end{Shaded}

\begin{verbatim}
# [1] 1
# [1] 2
# [1] 3
# [1] 4
\end{verbatim}
\end{frame}

\begin{frame}[fragile]{Loops en R}
\protect\hypertarget{loops-en-r-1}{}
Componentes de un loop:

\textbf{for} = función que identifica los procedimientos del loop.

\textbf{in} = especificación del objeto (vector, factor, matriz) sobre
el que se llevarán a cabo las iteraciones.

\textbf{()} = argumentos de unidades \emph{i} o \emph{j} de la función.

\textbf{\{\}} = operaciones de la función sobre cada \emph{i} o \emph{j}

\begin{Shaded}
\begin{Highlighting}[]
\CommentTok{\#Sintaxis}
\ControlFlowTok{for}\NormalTok{ (i }\ControlFlowTok{in} \DecValTok{1}\SpecialCharTok{:}\DecValTok{4}\NormalTok{)\{}
\FunctionTok{print}\NormalTok{(i }\SpecialCharTok{\^{}} \DecValTok{2}\NormalTok{)  }\CommentTok{\# elevamos al cuadrado cada "i" que estamos iterando}
\NormalTok{\}}
\end{Highlighting}
\end{Shaded}

\begin{verbatim}
# [1] 1
# [1] 4
# [1] 9
# [1] 16
\end{verbatim}
\end{frame}

\begin{frame}[fragile]{Loops en R}
\protect\hypertarget{loops-en-r-2}{}
Note que las especificaciones de un loop pueden ser aplicadas para la
creación de distintos objetos, como es el caso de una matriz con filas
\emph{i} y columnas \emph{j}. Por otro lado, se pueden definir vectores
para integrarlos en los loops. En este sentido, losloop son muy
flexibles respecto de lo que pueden realizar, ya sea con números o
palabras

\begin{Shaded}
\begin{Highlighting}[]
\CommentTok{\# Sintaxis para vectores}

\NormalTok{perros }\OtherTok{\textless{}{-}} \FunctionTok{c}\NormalTok{(}\StringTok{"Naruto"}\NormalTok{, }\StringTok{"Chopin"}\NormalTok{, }\StringTok{"Yeti"}\NormalTok{, }\StringTok{"sultan"}\NormalTok{, }\StringTok{"fido"}\NormalTok{, }\StringTok{"yonofui"}\NormalTok{)}

\ControlFlowTok{for}\NormalTok{ (i }\ControlFlowTok{in} \DecValTok{1}\SpecialCharTok{:}\FunctionTok{length}\NormalTok{(perros)) \{}
  \FunctionTok{print}\NormalTok{(}\FunctionTok{paste}\NormalTok{(}\StringTok{"Mi perro se llamaba:"}\NormalTok{, perros[i]))}
\NormalTok{\}}
\end{Highlighting}
\end{Shaded}

\begin{verbatim}
# [1] "Mi perro se llamaba: Naruto"
# [1] "Mi perro se llamaba: Chopin"
# [1] "Mi perro se llamaba: Yeti"
# [1] "Mi perro se llamaba: sultan"
# [1] "Mi perro se llamaba: fido"
# [1] "Mi perro se llamaba: yonofui"
\end{verbatim}
\end{frame}

\begin{frame}[fragile]{Loops en R}
\protect\hypertarget{loops-en-r-3}{}
Mediante loops es posible obtener resultados de promedios, medianas u
otras operaciones que necesitemos de nuestros datos. Esto puede ser de
mucha ayuda en la presencia bases de datos más grandes dónde, por
ejemplo, sacar un promedio para cada valor es imposible.

\begin{Shaded}
\begin{Highlighting}[]
\CommentTok{\# Usando la librería tidyverse generamos datos aleatorios en formato tibble}
\NormalTok{df }\OtherTok{\textless{}{-}} \FunctionTok{tibble}\NormalTok{(}
  \AttributeTok{a =} \FunctionTok{rnorm}\NormalTok{(}\DecValTok{3}\NormalTok{),}
  \AttributeTok{b =} \FunctionTok{rnorm}\NormalTok{(}\DecValTok{3}\NormalTok{),}
  \AttributeTok{c =} \FunctionTok{rnorm}\NormalTok{(}\DecValTok{3}\NormalTok{),}
  \AttributeTok{d =} \FunctionTok{rnorm}\NormalTok{(}\DecValTok{3}\NormalTok{))}

\CommentTok{\#resivamos los datos}
\NormalTok{df}
\end{Highlighting}
\end{Shaded}

\begin{verbatim}
## # A tibble: 3 x 4
##        a      b      c       d
##    <dbl>  <dbl>  <dbl>   <dbl>
## 1 -1.47  -1.67  -0.719  1.22  
## 2  0.617  0.455  1.08  -0.994 
## 3  0.725 -0.710 -0.346  0.0339
\end{verbatim}
\end{frame}

\begin{frame}[fragile]{Loops en R}
\protect\hypertarget{loops-en-r-4}{}
\begin{Shaded}
\begin{Highlighting}[]
\CommentTok{\#Sacar resultados 1 por 1 }
\FunctionTok{median}\NormalTok{(df}\SpecialCharTok{$}\NormalTok{a)}
\end{Highlighting}
\end{Shaded}

\begin{verbatim}
# [1] 0.6171565
\end{verbatim}

\begin{Shaded}
\begin{Highlighting}[]
\FunctionTok{median}\NormalTok{(df}\SpecialCharTok{$}\NormalTok{b)}
\end{Highlighting}
\end{Shaded}

\begin{verbatim}
# [1] -0.7102614
\end{verbatim}

\begin{Shaded}
\begin{Highlighting}[]
\FunctionTok{median}\NormalTok{(df}\SpecialCharTok{$}\NormalTok{c)}
\end{Highlighting}
\end{Shaded}

\begin{verbatim}
# [1] -0.3463874
\end{verbatim}

\begin{Shaded}
\begin{Highlighting}[]
\FunctionTok{median}\NormalTok{(df}\SpecialCharTok{$}\NormalTok{d)}
\end{Highlighting}
\end{Shaded}

\begin{verbatim}
# [1] 0.03386827
\end{verbatim}

\begin{Shaded}
\begin{Highlighting}[]
\CommentTok{\# Loops para los mismos resultados}
\NormalTok{resultados }\OtherTok{\textless{}{-}} \FunctionTok{vector}\NormalTok{(}\StringTok{"double"}\NormalTok{, }\FunctionTok{ncol}\NormalTok{(df))  }\CommentTok{\#  output}
\ControlFlowTok{for}\NormalTok{ (i }\ControlFlowTok{in} \FunctionTok{seq\_along}\NormalTok{(df)) \{            }\CommentTok{\#  secuencia}
\NormalTok{  resultados[[i]] }\OtherTok{\textless{}{-}} \FunctionTok{median}\NormalTok{(df[[i]])      }\CommentTok{\#  cuerpo}
\NormalTok{\}}
\NormalTok{resultados}
\end{Highlighting}
\end{Shaded}

\begin{verbatim}
# [1]  0.61715648 -0.71026144 -0.34638739  0.03386827
\end{verbatim}
\end{frame}

\begin{frame}[fragile]{Loops en R}
\protect\hypertarget{loops-en-r-5}{}
el comando \emph{if} en loops permite colocar condiciones dentro de las
operaciones que estamos realizando. Esto puede ser muy útil, por
ejemplo, para definir dónde queremos realizar ciertas operaciones dentro
de una base de datos. Veamos un ejemplo:

\begin{Shaded}
\begin{Highlighting}[]
\NormalTok{x }\OtherTok{\textless{}{-}} \DecValTok{5}
\ControlFlowTok{if}\NormalTok{(x }\SpecialCharTok{\textgreater{}} \DecValTok{0}\NormalTok{)\{ }\CommentTok{\#condición}
\FunctionTok{print}\NormalTok{(}\StringTok{"número positivo"}\NormalTok{)}
\NormalTok{\}}
\end{Highlighting}
\end{Shaded}

\begin{verbatim}
# [1] "número positivo"
\end{verbatim}

Acá la expresión del código es ``verdadera'' si la sentencia se ejecuta,
dada la condición que colocamos incialmente \emph{``x \textgreater{}
0''}, por lo que generara una respuesta en ``character'' que dice
``número positivo'', lo cual definimos en el output.
\end{frame}

\begin{frame}[fragile]{Loops en R}
\protect\hypertarget{loops-en-r-6}{}
Veamos un ejemplo más complejo. Supongamos que Juan tiene las siguientes
notas en distintas asignaturas: 3,6,2,1,5,7 y queremos agregar una
columna de aprobado y reprobado según la asignatura.

\begin{Shaded}
\begin{Highlighting}[]
\NormalTok{class }\OtherTok{\textless{}{-}} \FunctionTok{c}\NormalTok{(}\DecValTok{3}\NormalTok{,}\DecValTok{6}\NormalTok{,}\DecValTok{2}\NormalTok{,}\DecValTok{1}\NormalTok{,}\DecValTok{5}\NormalTok{,}\DecValTok{7}\NormalTok{)  }\CommentTok{\# notas de distintos ramos}
\NormalTok{p }\OtherTok{\textless{}{-}} \FunctionTok{c}\NormalTok{()}
\ControlFlowTok{for}\NormalTok{ (i }\ControlFlowTok{in} \DecValTok{1}\SpecialCharTok{:}\FunctionTok{length}\NormalTok{(class))\{}
\ControlFlowTok{if}\NormalTok{(class[i] }\SpecialCharTok{\textgreater{}=} \DecValTok{4}\NormalTok{) p[i] }\OtherTok{=} \StringTok{"aprobado"}
\ControlFlowTok{if}\NormalTok{(class[i] }\SpecialCharTok{\textless{}} \DecValTok{4}\NormalTok{) p[i] }\OtherTok{=} \StringTok{"reprobado"}
\NormalTok{\}}
\FunctionTok{as.data.frame}\NormalTok{(}\FunctionTok{rbind}\NormalTok{(p,class))}
\end{Highlighting}
\end{Shaded}

\begin{verbatim}
#              V1       V2        V3        V4       V5       V6
# p     reprobado aprobado reprobado reprobado aprobado aprobado
# class         3        6         2         1        5        7
\end{verbatim}

Donde los componentes \emph{if}:

\textbf{if} = función de especificación lógica.

\textbf{()} = condiciones lógicas para unidades \emph{i} o \emph{j} de
la función.

\textbf{\{\}} = operaciones de la función \emph{if} sobre cada \emph{i}
o \emph{j}
\end{frame}

\begin{frame}[fragile]{Medidas de asociación}
\protect\hypertarget{medidas-de-asociaciuxf3n}{}
Trabajaremos las medidas de asociación utilizando la base de datos del
Observatoriod de conflictos (OCS) de COES. Esta base de datos es pública
y la hemos estado utilizando en ayudantías anteriores. La base de datos
está previamente trabajada para sus disintas variables, es decir, ha
sufrido recodificaciones y creación de variables.

\begin{Shaded}
\begin{Highlighting}[]
\FunctionTok{head}\NormalTok{(df1)}
\end{Highlighting}
\end{Shaded}

\begin{verbatim}
# # A tibble: 6 x 16
#   ano   region     educa~1 indig~2 laboral salud pacif~3 disru~4 viole~5 organ~6
#   <fct> <fct>      <fct>   <fct>   <fct>   <fct> <fct>   <fct>   <fct>   <fct>  
# 1 2009  Metropoli~ No      No      No      No    Sí      No      No      1 orga~
# 2 2009  Tarapacá   Sí      No      No      No    Sí      No      No      1 orga~
# 3 2009  Tarapacá   No      No      Sí      No    Sí      Sí      No      1 orga~
# 4 2009  O´Higgins  No      No      Sí      No    No      Sí      No      Sin or~
# 5 2009  Araucanía  No      Sí      No      No    Sí      No      No      Sin or~
# 6 2009  Araucanía  No      No      No      No    No      Sí      No      Sin or~
# # ... with 6 more variables: nacional <fct>, macrozona <chr>,
# #   estudiantes <fct>, trabajadores <fct>, ppolicial <fct>, apolicial <fct>,
# #   and abbreviated variable names 1: educacion, 2: indigena, 3: pacifica,
# #   4: disruptiva, 5: violenta, 6: organizacion
# # i Use `colnames()` to see all variable names
\end{verbatim}
\end{frame}

\begin{frame}{Medidas de asociación}
\protect\hypertarget{medidas-de-asociaciuxf3n-1}{}
Contexto:

En la última década han existido desarrollos teóricos importantes para
aclarar el significado de la represión policial y para recentrar su
estudio en torno a nociones más amplias de ``control de la protesta'', o
``el control social de la protesta''. En relación con esto último, Earl
(2004) sostiene que parte relevante de la investigación sobre el control
de la protesta estudia la represión como forma característica de
observar las conductas que tienen las policías en escenarios de
movilización social (Koopman, 1995; Kriesi et al, 1995; Mc Adams, 1982,
Davenport. 2007). Este tipo aproximaciones basadas en las acciones
represivas de las policías para mantener el orden público muchas veces
oscurecen interrogantes importantes respecto de la heterogeneidad de los
actores y conductas policiales empleadas. Por un lado, nublan la
posibilidad de observar otras acciones que no involucran el uso dela
coerción como estrategia de disuasión o control del orden público. Por
otra parte, hablar de control dela protesta permite desmontar la idea de
que la represión es un sinónimo de violencia estatal y esclarecerel arco
de posibilidades que existen en las conductas policiales durante las
manifestaciones.

En esta ocación estaremos trabajando sobre las medidas de asociación a
través de la relación entre acción policial y tácticas o repertorios de
la protesta. Esto datos podemos encontrarlos para distintas regiones y
distintos años entre el 2009-2019. Para efectos del análisis
utilizaremos la variable \textbf{acción} policial, la cuál fue creada a
partir de distintas variables que contiene la base de datos del OCS
(enfrentamientos directos con manifestantes, uso de carros lanzagua o
gases lacrimógenos, uso de armas de fuego, uso de detenciónd de
manifestantes, solo presencia policial de la protesta).
\end{frame}

\begin{frame}[fragile]{Medidas de asociación}
\protect\hypertarget{medidas-de-asociaciuxf3n-2}{}
\begin{Shaded}
\begin{Highlighting}[]
\CommentTok{\# Presencia policial 2009{-}2019}
\FunctionTok{table}\NormalTok{(df1}\SpecialCharTok{$}\NormalTok{ppolicial) }\CommentTok{\# 26\% y 74\%}
\end{Highlighting}
\end{Shaded}

\begin{verbatim}
# 
#    Si    No 
#  6079 17318
\end{verbatim}

\begin{Shaded}
\begin{Highlighting}[]
\CommentTok{\# Acción policial 2009{-}2019 }
\FunctionTok{table}\NormalTok{(df1}\SpecialCharTok{$}\NormalTok{apolicial) }\CommentTok{\# 44\% y 56\%}
\end{Highlighting}
\end{Shaded}

\begin{verbatim}
# 
#  Control negociado Violencia Policial 
#               2697               3382
\end{verbatim}

\begin{Shaded}
\begin{Highlighting}[]
\CommentTok{\# Tácticas de la protesta  2009{-}2019 }
\FunctionTok{table}\NormalTok{(df1}\SpecialCharTok{$}\NormalTok{violenta) }\CommentTok{\# 45\% y 55\%}
\end{Highlighting}
\end{Shaded}

\begin{verbatim}
# 
#    No    Sí 
# 19365  4033
\end{verbatim}
\end{frame}

\begin{frame}[fragile]{Medidas de asociación}
\protect\hypertarget{medidas-de-asociaciuxf3n-3}{}
Lo primero que haremos es ver la asociación que existe entre las
acciones policiales y las tácticas disruptivas durante una manifestación
en el espacio público. Esto revisado de forma transversal para los datos
que contiene el OCS para los años 2009-2019

\begin{Shaded}
\begin{Highlighting}[]
\CommentTok{\# Sintaxis}
\CommentTok{\#Relación bivariada entre presencia policial y táctica pacífica}

\NormalTok{ctable1 }\OtherTok{\textless{}{-}}\NormalTok{ df1 }\SpecialCharTok{\%\textgreater{}\%} \FunctionTok{with}\NormalTok{(}\FunctionTok{table}\NormalTok{(apolicial,disruptiva)) }\CommentTok{\# 2{-}way table}

\FunctionTok{print}\NormalTok{(ctable1)}
\end{Highlighting}
\end{Shaded}

\begin{verbatim}
#                     disruptiva
# apolicial              No   Sí
#   Control negociado  1248 1449
#   Violencia Policial 1365 2017
\end{verbatim}
\end{frame}

\begin{frame}[fragile]{Medidas de asociación}
\protect\hypertarget{medidas-de-asociaciuxf3n-4}{}
Recordatorio:

Tenemos independencia estadística si la ocurrencia de un evento no
afecta la probabilidad de la ocurrencia de otro evento. Dicho de otro
modo, la probabilidad de que ocurra \(y\) es independiente de qué valor
asume \(x\). Por lo tanto, nos encontramos frente a independencia
estadística si las probabilidades conjunta son iguales al producto de
sus probabilidades marginales \(P(XY) = P(X)P(Y)\). Las medidas de
asociación nos permitirán justamente poder evaluar este escenario.

\begin{Shaded}
\begin{Highlighting}[]
\CommentTok{\# Sintaxis}
\CommentTok{\#Relaciones multivariadas (en este caso bivariadas)}
\FunctionTok{prop.table}\NormalTok{(ctable1,}\DecValTok{1}\NormalTok{)}
\end{Highlighting}
\end{Shaded}

\begin{verbatim}
#                     disruptiva
# apolicial                   No        Sí
#   Control negociado  0.4627364 0.5372636
#   Violencia Policial 0.4036073 0.5963927
\end{verbatim}
\end{frame}

\begin{frame}[fragile]{Medidas de asociación}
\protect\hypertarget{medidas-de-asociaciuxf3n-5}{}
¿Cuál es la diferencia de proporciones \(\delta\) que se observa en la
acción policial (\emph{Y}) de acuerdo a la presencia de tácticas
disruptivas de los manifestantes (\emph{X}). Particularmente, la
proporción de un control violento de la protesta (\emph{PCN}) respecto
de un control negociado de la protesta (\emph{PCV}).

\begin{Shaded}
\begin{Highlighting}[]
\CommentTok{\# Sintaxis}
\CommentTok{\#Diferencia de proporciones}
\NormalTok{delta }\OtherTok{\textless{}{-}}\NormalTok{ (}\FloatTok{0.597{-}0.537}\NormalTok{)}
\NormalTok{delta}
\end{Highlighting}
\end{Shaded}

\begin{verbatim}
# [1] 0.06
\end{verbatim}

\textbf{R}= Existe una diferencia de proporción de 0.06 entre un control
violento de la protesta para tácticas disruptivas en comparación con un
control negociado de la protesta. Ahora, es importante recalcar que esto
es solo una diferencia de proporciones, es decir, aún no sabemos si esto
es estadísticamente significativo o no.
\end{frame}

\begin{frame}[fragile]{Medidas de asociación}
\protect\hypertarget{medidas-de-asociaciuxf3n-6}{}
\textbf{Odds Ratio}

La \emph{odds} basícamente se define como la razón entre éxito o
fracaso, es decir, la razón entre \(p\) y \(1-p\) ¿Cuáles son las odds
de un control violento de la protesta durante una manifestación
disruptiva?

\begin{Shaded}
\begin{Highlighting}[]
\FunctionTok{print}\NormalTok{(ctable1) }\CommentTok{\# utilizamos la primera tabla de contigencia}
\end{Highlighting}
\end{Shaded}

\begin{verbatim}
#                     disruptiva
# apolicial              No   Sí
#   Control negociado  1248 1449
#   Violencia Policial 1365 2017
\end{verbatim}

\begin{Shaded}
\begin{Highlighting}[]
\CommentTok{\# Sintaxis}
\CommentTok{\# Valor probabilístico de éxito}
\NormalTok{p }\OtherTok{\textless{}{-}}\NormalTok{ (}\DecValTok{2017}\SpecialCharTok{/}\NormalTok{(}\DecValTok{1365} \SpecialCharTok{+}\DecValTok{2017}\NormalTok{))}

\CommentTok{\# Odds (P/1{-}P)}
\NormalTok{odds }\OtherTok{\textless{}{-}}\NormalTok{ (p}\SpecialCharTok{/}\NormalTok{(}\DecValTok{1}\SpecialCharTok{{-}}\NormalTok{p)) }
\NormalTok{odds}
\end{Highlighting}
\end{Shaded}

\begin{verbatim}
# [1] 1.477656
\end{verbatim}

\textbf{R}:Las odds (``chances'') de que existe un control violento de
la protesta durante una manifestación disruptiva son de 1.47
\end{frame}

\begin{frame}{Medidas de asociación}
\protect\hypertarget{medidas-de-asociaciuxf3n-7}{}
\textbf{Odds ratio}

A partir de la tabla de contigencia que tenemos podemos medir la
asociación entre variables, es decir las \emph{odds ratio}.

Ahora bien, en una tabla de 2x2 la razón de odds \(\theta\) es la razón
de éxito en dos filas, o
\(\theta= \frac{odds1}{odds2}=\frac{P_1/(1-P_1)}{P_2/(1-P_2)}\)

Sabemos que\ldots{}

\begin{itemize}
\item
  Si \(\theta=1\) hay igualdad de odds (``chances'') y, por lo tanto,
  hay independencia entre variables.
\item
  Si \(\theta > 1\) entonces el éxito es más probable para el grupo en
  el numerador.
\item
  Si \(\theta < 1\) entonces el éxito es más probable para el grupo en
  el denominador.
\end{itemize}
\end{frame}

\begin{frame}[fragile]{Medidas de asociación}
\protect\hypertarget{medidas-de-asociaciuxf3n-8}{}
Siempre es relevante, antes de calcular el \emph{odds ratio}, saber que
queremos calcular o que pregunta nos estamos haciendo. Por ejemplo:

¿Cuál es la razón de odds de un control negociado de la protesta en
presencia de tácticas disruptivas?.

\begin{Shaded}
\begin{Highlighting}[]
\FunctionTok{prop.table}\NormalTok{(ctable1,}\DecValTok{1}\NormalTok{)}
\end{Highlighting}
\end{Shaded}

\begin{verbatim}
#                     disruptiva
# apolicial                   No        Sí
#   Control negociado  0.4627364 0.5372636
#   Violencia Policial 0.4036073 0.5963927
\end{verbatim}

\begin{Shaded}
\begin{Highlighting}[]
\CommentTok{\# Sintaxis}
\CommentTok{\# Razón de Odds en proporciones }
\NormalTok{OR }\OtherTok{\textless{}{-}}\NormalTok{ ((}\FloatTok{0.537}\SpecialCharTok{/}\FloatTok{0.462}\NormalTok{)}\SpecialCharTok{/}\NormalTok{(}\FloatTok{0.596}\SpecialCharTok{/}\FloatTok{0.403}\NormalTok{)) }
\NormalTok{OR}
\end{Highlighting}
\end{Shaded}

\begin{verbatim}
# [1] 0.7859431
\end{verbatim}
\end{frame}

\begin{frame}[fragile]{Medidas de asociación}
\protect\hypertarget{medidas-de-asociaciuxf3n-9}{}
\begin{Shaded}
\begin{Highlighting}[]
\FunctionTok{print}\NormalTok{ (ctable1)}
\end{Highlighting}
\end{Shaded}

\begin{verbatim}
#                     disruptiva
# apolicial              No   Sí
#   Control negociado  1248 1449
#   Violencia Policial 1365 2017
\end{verbatim}

\begin{Shaded}
\begin{Highlighting}[]
\CommentTok{\# esto también se puede calcular como producto cruzado }

\NormalTok{theta }\OtherTok{=}\NormalTok{ ((}\DecValTok{1449}\SpecialCharTok{*}\DecValTok{1365}\NormalTok{)}\SpecialCharTok{/}\NormalTok{(}\DecValTok{1248}\SpecialCharTok{*}\DecValTok{2017}\NormalTok{)) }\CommentTok{\# Cross{-}product Ratio}
\NormalTok{theta}
\end{Highlighting}
\end{Shaded}

\begin{verbatim}
# [1] 0.7857431
\end{verbatim}

\begin{Shaded}
\begin{Highlighting}[]
\NormalTok{theta\_p }\OtherTok{\textless{}{-}}\NormalTok{ (theta}\DecValTok{{-}1}\NormalTok{)}\SpecialCharTok{*}\DecValTok{100}
\NormalTok{theta\_p}
\end{Highlighting}
\end{Shaded}

\begin{verbatim}
# [1] -21.42569
\end{verbatim}

\textbf{R}: Las odds de la existencia de un control negociado de la
protesta son 0.8 veces las odds de un control violento de la protesta
cuándo existen tácticas disruptivas, es decir, el control negociado de
la protesta es un 21 \% más bajo que el control violento de este tipo de
manifestaciones con presencia de tácticas disruptivas.
\end{frame}

\begin{frame}[fragile]{Inferencia Estadística}
\protect\hypertarget{inferencia-estaduxedstica}{}
Continuando con nuestros cálculos previos\ldots{}

\textbf{Intervalos de confianza}

¿Es posible afirmar que nuestro valor \(\delta= 0.06\), que observa en
la muestra una diferencia de proporciones, es estadísticamente
significativo a un 99\% de confianza?

\begin{Shaded}
\begin{Highlighting}[]
\CommentTok{\# Sintaxis}
\CommentTok{\#CI Diferencia de proporciones}
\NormalTok{PCV  }\OtherTok{\textless{}{-}} \FloatTok{0.5963927} \CommentTok{\# control violento}
\NormalTok{PCN  }\OtherTok{\textless{}{-}} \FloatTok{0.5372636} \CommentTok{\# Control negociado}
\NormalTok{se }\OtherTok{\textless{}{-}}  \FunctionTok{sqrt}\NormalTok{((PCV}\SpecialCharTok{*}\NormalTok{(}\DecValTok{1} \SpecialCharTok{{-}}\NormalTok{ PCV))}\SpecialCharTok{/}\DecValTok{4690} \SpecialCharTok{+}\NormalTok{  (PCN}\SpecialCharTok{*}\NormalTok{(}\DecValTok{1} \SpecialCharTok{{-}}\NormalTok{ PCN))}\SpecialCharTok{/}\DecValTok{3919}\NormalTok{)}
\NormalTok{ci99\_delta }\OtherTok{\textless{}{-}}  \FunctionTok{c}\NormalTok{(}\AttributeTok{li=}\NormalTok{(delta }\SpecialCharTok{{-}} \FloatTok{2.58}\SpecialCharTok{*}\NormalTok{se), }\AttributeTok{ls=}\NormalTok{(delta }\SpecialCharTok{+} \FloatTok{2.58}\SpecialCharTok{*}\NormalTok{se)) }
\FunctionTok{print}\NormalTok{(ci99\_delta)}
\end{Highlighting}
\end{Shaded}

\begin{verbatim}
#         li         ls 
# 0.03236132 0.08763868
\end{verbatim}

\textbf{R}: La diferencia de proporciones observada es estadísticamente
significativa a un 99\% de confianza
\end{frame}

\begin{frame}[fragile]{Inferencia Estadística}
\protect\hypertarget{inferencia-estaduxedstica-1}{}
Test \(\chi^2\)

¿Es posible rechazar la \(H_0\) que afirma la independencia estadística
de la relación bivariada entre la presencia de tácticas disruptivas y el
tipo de control policial de la protesta? \emph{Ojo}: Recordemos que en
tablas de 2x2 la independencia estadística entre variables equivale a
\(H_0: \pi_{ij}=\pi_{i+}\pi_{+j}\), con una posible
\(H_1: \pi_{ij}\neq\pi_{i+}\pi_{+j}\) y que los grados de libertad están
determinados por \(df= (i-1)(j-1)\).

\begin{Shaded}
\begin{Highlighting}[]
\CommentTok{\# Sintaxis}
\CommentTok{\# Test Chi2 de independencia estadística}
\FunctionTok{chisq.test}\NormalTok{(ctable1,}\AttributeTok{correct =} \ConstantTok{FALSE}\NormalTok{)}
\end{Highlighting}
\end{Shaded}

\begin{verbatim}
# 
#   Pearson's Chi-squared test
# 
# data:  ctable1
# X-squared = 21.405, df = 1, p-value = 3.718e-06
\end{verbatim}

R= Con un valor \(\chi^2 = 21.405\) y un \(p=0.000003718\) es posible
rechazar \(H_0\) y afirmar que no existe independencia estadística entre
las variables en todos los niveles convencionales de confianza
\(p<0.001\), \(p<0.01\), \(p<0.05\). Hay asociación entre las variables.
\end{frame}

\end{document}
